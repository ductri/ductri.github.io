\documentclass[11pt,a4paper]{article}

% --------------------------------------------------------- 
% PACKAGES AND PREDEFINED SETTINGS
% --------------------------------------------------------- 

%\usepackage{lmodern}
%\usepackage{xcolor}
\usepackage{float}
\usepackage{caption}
\usepackage{subcaption}
%\usepackage[utf8x]{inputenc}
\usepackage[linesnumbered,lined,boxed,commentsnumbered,ruled,vlined]{algorithm2e}
\usepackage{enumitem}
%\usepackage[english]{babel}
\usepackage{amssymb}
\usepackage{amsmath}
\usepackage{amsthm}
\usepackage{graphicx} % Allows including images
\usepackage[colorlinks=false]{hyperref} % the option is there to remove the square around links which is what I don't like.
\usepackage{perpage} 
\MakePerPage{footnote} % Reset the footnote counter perpage. may require to run latex twice.
\usepackage[margin=2cm]{geometry} % This is here to fit more text into the page.
\setlength{\parindent}{0pt} % No indentation for paragraphs. Because that is just old.
%\setlength{\parskip}{\baselineskip} % Instead use vertical paragraph spacing.
\fontencoding{T1} % the better font encoding.
\usepackage{chngcntr} % https://tex.stackexchange.com/questions/28333/continuous-v-per-chapter-section-numbering-of-figures-tables-and-other-docume
%\usepackage{thmtools} % http://ftp.math.purdue.edu/mirrors/ctan.org/macros/latex/exptl/thmtools/thmtools.pdf
\usepackage{import}
\usepackage{pdfpages}
\usepackage{transparent}
\usepackage{xcolor}

\usepackage{listings}
\lstset{language=python}
\definecolor{codegreen}{rgb}{0,0.6,0}
\definecolor{codegray}{rgb}{0.5,0.5,0.5}
\definecolor{codepurple}{rgb}{0.58,0,0.82}
\definecolor{backcolour}{rgb}{0.95,0.95,0.92}

\lstdefinestyle{mystyle}{
    backgroundcolor=\color{backcolour},   
    commentstyle=\color{codegreen},
    keywordstyle=\color{magenta},
    numberstyle=\tiny\color{codegray},
    stringstyle=\color{codepurple},
    basicstyle=\ttfamily\footnotesize,
    breakatwhitespace=false,         
    breaklines=true,                 
    captionpos=b,                    
    keepspaces=true,                 
    numbers=left,                    
    numbersep=5pt,                  
    showspaces=false,                
    showstringspaces=false,
    showtabs=false,                  
    tabsize=2
}
\lstset{style=mystyle}
\usepackage[
backend=biber,
style=alphabetic,
citestyle=alphabetic
]{biblatex}
%\addbibresource{citation.bib}
\usepackage{afterpage}
\usepackage{multirow}

% --------------------------------------------------------- 
% SETTINGS
% --------------------------------------------------------- 
\counterwithin{figure}{section}
\newcommand\mycommfont[1]{\footnotesize\ttfamily\textcolor{blue}{#1}}
\SetCommentSty{mycommfont}
% --------------------------------------------------------- 
% CUSTOM COMMANDS
% --------------------------------------------------------- 

\def\green{\color{green}}
\def\red{\color{red}}
\def\blue{\color{blue}}
\newcommand{\rank}[1]{\text{rank}(#1)}
\newcommand{\pr}[1]{\text{Pr}\left(#1\right)}
\newcommand{\st}{\text{subject to}\quad }
\newcommand{\trace}[1]{\text{tr}\left(#1\right)}
\newcommand{\norm}[1]{\left\lVert#1\right\rVert}
\newcommand{\abs}[1]{\left\lvert#1\right\rvert}
\newcommand{\vect}[1]{\text{vec}\left(#1\right)}
\newcommand{\diag}[1]{\text{diag}\left(#1\right)}
\newcommand\numberthis{\addtocounter{equation}{1}\tag{\theequation}} % https://tex.stackexchange.com/questions/42726/align-but-show-one-equation-number-at-the-end 
\newcommand{\set}[1]{\left\{#1\right\}}
\newcommand*\diff{\mathop{}\!\mathrm{d}}

\newcommand{\incfig}[2][1]{%
    \def\svgwidth{#1\columnwidth}
    \import{./figures/}{#2.pdf_tex}
}

\DeclareMathOperator*{\argmax}{arg\,max}
\DeclareMathOperator*{\argmin}{arg\,min}
\DeclareMathOperator*{\minimize}{minimize}
\DeclareMathOperator*{\maximize}{maximize}

% --------------------------------------------------------- 
% CUSTOM ENVIRONMENTS
% --------------------------------------------------------- 
\theoremstyle{plain}
\newtheorem{theorem}{Theorem}[section]
\newtheorem{lemma}{Lemma}[section]

\theoremstyle{definition}
\newtheorem{definition}{Definition}[section]
\newtheorem{corollary}{Corollary}[theorem]

\theoremstyle{remark}
\newtheorem{remark}{Remark}[section]

% --------------------------------------------------------- 
% TITLE, AUTHORS, ...
% --------------------------------------------------------- 
\title{Tensor Notes}
\author{	Tri Nguyen \\
        \texttt{nguyetr9@oregonstate.edu} \\
        }

% --------------------------------------------------------- 
% BEGIN DOCUMENT
% --------------------------------------------------------- 
\begin{document}
\maketitle

\section{Useful identities}%
\label{sec:usefull_identities}
\begin{align}
    &\text{If } \mathcal{Y} = [\mathbf{A}, \mathbf{B}, \mathbf{C}] \Rightarrow \mathcal{Y} \times_{1} \mathbf{P}_1 \times_{2} \mathbf{P}_2 \times_{3} \mathbf{P}_3 = [\mathbf{P}_1 \mathbf{A}, \mathbf{P}_2 \mathbf{B}, \mathbf{P}_3 \mathbf{C}] \label{eq:tensor_mode_product} \\
    &(\mathbf{A} \odot \mathbf{B})^T (\mathbf{A} \odot  \mathbf{B}) = (\mathbf{A}^T \mathbf{A}) * (\mathbf{B}^T \mathbf{B}) \label{eq:khatrirao_hadamard}\\
    & (\mathbf{A} \otimes \mathbf{B}) (\mathbf{C} \otimes \mathbf{D}) = (\mathbf{A}\mathbf{C}) \otimes (\mathbf{C} \mathbf{D}) \\
    &\text{vec}(\mathbf{A} \mathbf{X}\mathbf{B}^T ) =  (\mathbf{B} \otimes \mathbf{A}) \text{vec}(\mathbf{X}) \\
    &\dfrac{\partial\; \norm{\mathbf{A}\mathbf{X}\mathbf{B} - \mathbf{C}}_F^2}{\partial\; \mathbf{X}} 
    = \mathbf{A}^T \mathbf{A} \mathbf{X} \mathbf{B}\mathbf{B}^T  - \mathbf{A}^T \mathbf{C} \mathbf{B}^T  \label{eq:derivative_AXB} \\
    &\text{function } \texttt{reshape} \text{ in Matlab is \ldots }
\end{align}
Matrisized tensor: This is just 1 special treameat to rearrage a tensor to a matrix. Mode-n matricization of of tensor $\underline{\mathbf{X}} \in \mathbb{R}^{I_1 \times \ldots  \times I_N}$, denoted as $\underline{\mathbf{X}}^{(n)}$ is a matrix forming by arranging mode-n fibers as rows. Note that this description is \textit{not} sufficient since it doesnt specify the row ordering. Anyway, we do use a consistent matricization.

Suppose $\underline{\mathbf{X}}=[\mathbf{A}_1, \ldots , \mathbf{A}_N]$ then 
\begin{align*}
    \underline{\mathbf{X}}^{(n)} &:= \left(  \odot_{i=N, i\neq n}^{1} \mathbf{A}_i\right) \mathbf{A}_n^{\rm T}  \\
    &= \left( \mathbf{A}_N \odot \mathbf{A}_{N-1} \odot \ldots \odot \mathbf{A}_{n+1} \odot \mathbf{A}_{n-1} \odot \ldots \mathbf{A}_{1} \right) \mathbf{A}_n^{\rm T} 
\end{align*}
which is implemented as 
\begin{itemize}
    \item In python, 
    \item With tensorlab, $\underline{\mathbf{X}}^{(n)} = \texttt{tens2mat($\underline{\mathbf{X}}$, n)'}$
    \item In matlab, $\underline{\mathbf{X}}^{(n)} = \texttt{reshape(permute($\underline{\mathbf{X}}, [n, 1, 2, \ldots, n-1, n+1, \ldots , N]$), $I_N$, [])'}$
\end{itemize}
And vectorization operator,
\begin{align*}
\text{vect}(\underline{\mathbf{X}}) 
&:= \left(  \odot_{i=N}^{1} \mathbf{A}_i\right) \mathbf{1}_F \\
&= (\mathbf{A}_N \odot \ldots \odot \mathbf{A}_1) \mathbf{1}_F \\
&= \text{vec}(\mathbf{A}_1 \mathbf{I}_F (\mathbf{A}_N  \odot \ldots \odot \mathbf{A}_2)^{\rm T}) \\
&= \text{vec}(\mathbf{A}_1 (\mathbf{A}_N  \odot \ldots \odot \mathbf{A}_2)^{\rm T}) \\
&= \text{vec}(\underline{\mathbf{X}}^{(1)}) \\
\end{align*} 

\begin{proof}[Proof of \ref{eq:derivative_AXB}]
    Since 
    \[
        f(\mathbf{X}) :=
    \norm{\mathbf{A}\mathbf{X}\mathbf{B} - \mathbf{C}}_F^2 
    = \norm{\text{vec}(\mathbf{A}\mathbf{X}\mathbf{B}) - \text{vec}(\mathbf{C})}_2^2
    = \norm{(\mathbf{B}^T \otimes \mathbf{A}) \text{vec}(\mathbf{X}) - \text{vec}(\mathbf{C})}_2^2
    \] 
    hence,
    \begin{align*}
        \dfrac{\partial\; f(\mathbf{X})}{\partial\; \text{vec}(\mathbf{X})}
        &= (\mathbf{B}^T  \otimes \mathbf{A})^T \left( (\mathbf{B}^T \otimes \mathbf{A}) \text{vec}(\mathbf{X}) - \text{vec}(\mathbf{C}) \right) \\
        &= (\mathbf{B} \mathbf{B}^T ) \otimes (\mathbf{A}^T \mathbf{A}) \text{vec}(\mathbf{X}) - (\mathbf{B} \otimes \mathbf{A}^T ) \text{vec}(\mathbf{C}) \\
        &= \text{vec}(\mathbf{A}^T \mathbf{A} \mathbf{X} \mathbf{B}\mathbf{B}^T - \text{vec}(\mathbf{A}^T \mathbf{C} \mathbf{B}^T )) \\
        \Rightarrow 
        \dfrac{\partial\; f(\mathbf{X})}{\partial\; \mathbf{X}} &= \mathbf{A}^T \mathbf{A}\mathbf{X}\mathbf{B}\mathbf{B}^T - \mathbf{A}^T \mathbf{C}\mathbf{B}^T 
    \end{align*}
\end{proof}



\end{document}
