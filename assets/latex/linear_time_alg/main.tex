\documentclass[11pt,a4paper]{article}
\usepackage{tri_preamble}

% --------------------------------------------------------- 
% TITLE, AUTHORS, ...
% --------------------------------------------------------- 
\title{My Observation on Common Pattern of Linear Time Algorithms}
\author{	Tri Nguyen \\\\
        \texttt{nguyetr9@oregonstate.edu} \\\\
        }
% --------------------------------------------------------- 
% BEGIN DOCUMENT
% --------------------------------------------------------- 
\begin{document}
\maketitle
As preparing for coding interview, I have been practicing solving coding problems. It was struggling to me many times to come up with solution within reasonable time. Then, I think there should be a better way to deal with these problems once and for all. So I came up with this unified view on how to solve a particular class of problems, i.e., problems of finding some value and they have optimal algorithms with O(n) running time and O(1) memory complexity.

\textit{Warning: This view has never been tested in real interview, so viewer discretion is advised. ;))}

\section{General Strategy}%
\label{sec:general_strategy}
Using recursion strategy to solve this problem:
Suppose we have solution for nums[0..k], how to produce solution for nums[0..k+1]? Or more generally, imagine that we have a ``state[k]'' assocciating to the input nums[0..k], then
\begin{itemize}
    \item how to deduce desired solution from ``state[k]''
    \item how to deduce the next ``state[k+1]'' associcating to input nums[0..k+1] based on the previous ``state[k]'' and the new element nums[k+1]
\end{itemize}
The requirement on state.
\begin{itemize}
    \item In term of memory, total memory to save state of step should be O(1). The only way to realize this is to have a fixed amount of memory for "state".
    \item In term of computation, total computation cost of all step should be O(n). A trivial way is to compute O(1) for each step. Beside these trivial realization, we can distribute memory/computation unevenly among all steps, as long as the summation is O(n).
    \item I am not sure if there are other ways.
\end{itemize}

\section{Examples}%
\label{sec:examples}

\subsection{Find the maximum value}%
\label{sub:the_maximum_problem}

\subsection{The daily temperature problem}%
\label{sub:the_daily_temperature_problem}

\subsection{The maximum subarray problem}%
\label{sub:the_maximum_subarray_problem}




\end{document}

